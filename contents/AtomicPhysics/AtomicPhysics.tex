% Content: Atomic Physics
%------------------------

\section{Atomic Physics (10\%)}

It includes properties of electrons, Bohr's model, energy quantization, atomic structure, atomic spectra, selection rules, black-body radiation, X-rays, atoms in electric fields, and atoms in magnetic fields.
%%%%%%%%%%%%%%%%%%%%%%%%%%%%%%%%%%%%%%%%%%%%%%%%%%%%%%%%%%%%%%%%%%%%%%%%%%%%%%

If an external electron strikes and knocks the innermost electron then, another electron will jump into that place such occurs K (from some Bohr level $n \geq 2 \to n = 1$), L ($n \geq 2 \to n = 2$), M($n \geq 2 \to n = 3$), L ($n \geq 2 \to n = 4$) series transition. For the specific transition, the minimum transition \emph{energy} is given by \emph{Moseley's law},
\begin{align}
E = R_{e} (Z - \beta)^{2} \left( \frac{1}{n_{\text{final}}^{2}} - \frac{1}{n_{\text{initial}}^{2}} \right)
\end{align}
where, $R_{e}$ is the Rydberg's energy (or constant) which is equal to 13.6 eV. Because the ground state energy of Hydrogen atom is referred as 1 Rydberg. And Rydberg's constant is mass dependent. i.e.
\begin{align*}
R_{\text{hydrogen}} = \frac{m_{e} m_{p}}{m_{e} + m_{p}} \frac{e^{4}}{8c\epsilon^{2} h^{3}}
\end{align*}

The ratio of the mass of proton to electron is 1836:1.

Atoms with a single electron in its outer shell can be thought of as analogous to a hydrogen atom and it has the electron charge distribution to be spherically symmetrical.

From the Bohr model, we get the approximations of any hydrogen like atoms as 
\begin{align*}
E_{n} = -\frac{Z^{2}R_{e}}{n^{2}}
\end{align*}

Radioactive elements have very low binding energy in-compared to others. The standard deviation for radioactive emission is described by the Poisson noise, $\sigma = \sqrt{\lambda}$ where $\lambda$ is the average number of radioactive samples. 

Compton scattering is the phenomena by which photons collide with a particle, impart some of their kinetic energy to the particle and then scatters off at a lower energy. The compton equation is,
\begin{align}
\lambda_{final} - \lambda_{initial} = \frac{h}{m c}(1 - cos(\theta))
\end{align}
where $\lambda$ is the wavelength of photon, $m$ is its mass and $\theta$ is the scattering angle.

The Franck-Hertz experiment and related scattering experiments show us that at a specific energy range (4.9 volts to be specific in Franck-Hertz experiment) electrons begin to experience inelastic collisions and that energies lost in the collision came in discrete amounts.

The energy range for Photoelectric effect is measured in eV and that of Compton effect is in KeV, and pair production is in MeV.