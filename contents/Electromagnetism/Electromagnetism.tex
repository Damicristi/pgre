% Contents: Electromagnetism
%-----------------------------

\section{Electromagnetism (18\%)}

It includes electrostatics, current and DC circuits, magnetic fields in free space, Lorentz force, induction, Maxwell's equations and their applications, electromagnetic waves, AC circuits, and magnetic and electric fields in matter.

%%%%%%%%%%%%%%%%%%%%%%%%%%%%%%%%%%%%%%%%%%%%%%%%%%%%%%%%%%%%%%%%%%%%%%%%%%%%%%

Kirchhoff's voltage loop law says the sum of all voltages about a complete circuit must sum to zero. This is very important law when a circuit problem is given.

In a region where there are no charges inside but with a uniform potential on the surface then, the potential inside the center is exactly same that of surface.

The most ubiquitous Gauss's law is
\begin{align*}
\div\vec{E} = \frac{\rho}{\epsilon_{0}} 
\end{align*}
 where $\rho$ is the total electric charge density (charge per unit volume).
 
The exponent in Coulomb's inverse square law has been found to differ from two by less than one part in a billion by measuring the electric field inside a charged conducting shell and the experiment was Faraday's Ice Pail.

The shielding on the coaxial cable is to reduce the inteference with other electronic equipment so, there is no any presence of electromagnetic waves outside the cable.

If there is grounded conducting plane at the center $x = 0$ and a charge $q$ at a distance of $a$ then, there is also an image charge $-q$ which is at $-a$ from the center.

In capacitor, we have $Q = C V$ and its energy is $U = \frac{1}{2} C V^{2}$. For the circuit which follows ohms law, we have $V = I R$. Also,
\begin{align}
X = X_{0} e^{\frac{-t n}{RC}}
\end{align}
where $X$ can be $I$ or $V$ with $n = 1$, or $U$ with $n = 2$.