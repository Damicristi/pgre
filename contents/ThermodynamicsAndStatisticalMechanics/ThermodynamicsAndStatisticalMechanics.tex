% Content: Thermodynamics and Statistical Mechanics
%-----------------------------------------------------

\section{Thermodynamics and Statistical Mechanics (10\%)}

It includes the laws of thermodynamics, thermodynamic processes, equations of state, ideal gases, kinetic theory, ensembles, statistical concepts, calculation of thermodynamics quantities, thermal expansion, and heat transfer.

%%%%%%%%%%%%%%%%%%%%%%%%%%%%%%%%%%%%%%%%%%%%%%%%%%%%%%%%%%%%%%%%%%%%%%%%%%%%%%

In Carnot engine, the entropy of the system should be perfectly conserved. i.e. $dS = 0$. The maximum efficiency of a Carnot engine is,
\begin{align}
\eta = \frac{\text{Workdone}}{\text{Heat input}} = 1 - \frac{\text{Absolute temperature of cold reservoir}}{\text{Absolute temperature of hot reservoir}}
\end{align}

The mean free path equation for the probability of stopping a particle moving through medium is
\begin{align}
P(x) = n \sigma dx
\end{align}
where $P(x)$ is the probability of stopping the particle in the distance $dx$, n is the nuclei, $\sigma$ is the scattering cross section and $dx$ is the thickness of the scatterer.

According to Stefan's law (to be more specific, Stefan-Boltzmann's), power radiation of a blackbody is only dependent on temperature which is
\begin{align}
j = \sigma T^{4}
\end{align}
where $\sigma$ is a Stefan's constant. So, if you see a problem related to blackbody, remember that law.

Heat capacity of a molecule is determined by the number of degrees of freedom of the molecule. For example, in a mono-atomic gas, the heat capacity at constant volume is $C_{V} = \frac{3}{2}R$. For diatomic molecule connected by spring, we have additional degree of freedom due to vibration and rotation then, the heat capacity will be $\frac{3}{2}R + R + R = \frac{7}{2}R$.

