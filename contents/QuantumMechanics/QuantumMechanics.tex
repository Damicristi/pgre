% Content: Quantum Mechanics
%----------------------------

\section{Quantum Mechanics (12\%)}

It includes fundamental concepts, solutions of the Schr\"{o}dinger equation (like square wells, harmonic oscillators and hydrogen atoms), spin, angular momentum, wave function, symmetry, and elementary perturbation theory.

\subsection{Probablity of wave function}
The probability of finding a particle with wave function $\psi(x,t)$ within the volume\footnote{For one-dimension, volume is length of a line i.e. positions $(x, x+dx)$.} $dv$ is $|\psi(x,t)|^{2}dv$. So, the total probability $\int_{-\infty}^{\infty} |\psi(x,t)|^{2}dv = 1$. The wave function must be normalized and wave function should vanish exactly at $-\infty$ to $\infty$. For revision, we only consider one dimension most often.

\subsubsection{Dimension of wave function}
You can get its dimension by applying dimensional analysis in the total probability of wave function. i.e.
\begin{align}
\text{Dimension} = L^{-d/2}
\end{align}
where, d is the spatial dimensions. 

\subsection{Hermitian operator}
Hermitian is defined as a transpose of complex conjugate. An operator is said to be Hermitian operator if $\hat{A}^{\dagger} = (A^{*})^{T}= \hat{A}$ or to be more precise

\begin{align}
\int_{-\infty}^{\infty} f(x)^{*} (\hat{A} g(x)) dx = \int_{-\infty}^{\infty} (\hat{A} f(x))^{*} g(x) dx.
\end{align}
Here are some facts about Hermitian operator:
\begin{itemize}
\item All their eigenvalues are real.
\item Eigenfunctions corresponding to different eigenvalues are orthogonal. i.e.
\begin{align*}
\int_{-\infty}^{\infty} f(x)^{*} g(x) dx = 0. 
\end{align*}
\end{itemize}

\subsection{Observables in-term of Hermitian operator}
\begin{align*}
\langle\hat{A}\rangle = \int_{-\infty}^{\infty} \psi^{*} \hat{A} \psi dx
\end{align*}

In quantum mechanics, observables and operators are often interchangeable. An observable refer to a physical quantity whereas operator refers to its mathematical representation i.e. matrix.

\subsection{Dirac notation}
$\bra{a}$ refers to \emph{bra} and $\ket{b}$ refers to \emph{ket} which is as a whole refers to bra(c)ket notation or Dirac notation. The inner product of $\bra{a}$ and $\ket{b}$ is $\braket*{a}{b}$ where $\bra{a}$ is a row vector $a$ (also, a hermitian conjugate of $\ket{a}$), and $\ket{b}$ is column vector $b$. Just for knowledge not for pGRE, the state of a wave function can be modeled as a set of points in a Hilbert space which allows us to measure length and angle due to inner product. 

Here are some results of this notation that is good to remember. i.e.
\begin{itemize}
\item Complex conjugate of inner product, $\braket*{a}{b} = \braket*{b}{a}^{*}$. 
\item An operator $\hat{A}$ on $\ket{b}$, $\hat{A}\ket{b} = \ket*{\hat{A}b}$.
\item Hermitian conjugate of $\hat{A}$ i.e. $\hat{A}^{\dagger}$, $ \braket*{a}{\hat{A}b} =  \braket*{\hat{A}^{\dagger} a}{b} = \braket*{\hat{A} a}{b}$ because $\hat{A}^{\dagger} = \hat{A}$.
\item Function in the Dirac notation, $\braket*{x}{f} = f(x)$.
\item Inner product on function space, $\braket*{f}{g} = \int_{-\infty}^{\infty} f(x)^{*} g(x) dx$.
\item Observable $A$ in the Dirac notation, $\langle\hat{A}\rangle = \int_{-\infty}^{\infty} \psi^{*} \hat{A} \psi dx = \mel{\psi^{*}}{\hat{A}}{\psi}$.
\end{itemize}

\subsection{Quantum operators}
It is easy to remember quantum operators on a single axis than a general form.

\begin{align*}
\text{Position, } \hat{x} &= x \\ 
\text{Momentum, } \hat{p} &= -i\hbar\pdv{x} \\
\text{Kinetic energy, } \hat{T} &=  \frac{\hat{p}^{2}}{2m} = -\frac{\hbar^{2}}{2m} \pdv[2]{x}\\
\text{Hamiltonian, } \hat{H} &= \hat{T} + \hat{V} = -\frac{\hbar^{2}}{2m} \pdv[2]{x} + \hat{V}(x) \\
\text{Total Energy for time-dependent potential, } \hat{E} &= i\hbar \pdv{t}\\
\text{Total Energy for time-independent potential, } \hat{E} &= E \\
\text{Angular momentum, } \hat{L}_{x} &= -i\hbar \left( y\pdv{z} - z\pdv{y}\right)
\end{align*}
\begin{align*}
\hat{L}_{y} &= -i\hbar \left( z\pdv{x} - x\pdv{z}\right) \\
\hat{L}_{z} &= -i\hbar \left( x\pdv{y} - y\pdv{x}\right)
\end{align*}

\subsection{Schr\"odinger equation}
You can find the wave function from the solution of Schr\"o dinger equation.
\begin{align}
\hat{H} \psi(x,t) &= \hat{E} \psi(x,t)
\end{align}

\subsubsection{Time dependent}
\begin{align*}
\left(-\frac{\hbar^{2}}{2m} \pdv[2]{x} + \hat{V}(x)\right)\psi(x,t) &=  i\hbar \pdv{t}\psi(x,t)
\end{align*}

\subsubsection{Time independent}
\begin{align*}
\left(-\frac{\hbar^{2}}{2m} \pdv[2]{x} + \hat{V}(x)\right)\psi(x,t) &=  E\psi(x,t)
\end{align*}