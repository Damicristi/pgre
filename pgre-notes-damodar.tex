%%%%%%%%%%%%%%%%%%%%%%%%%%%%%%%%%%%%%%%%%%%%%%%%%%%%%%%
% Time-stamp: <2018-09-10 21:56:28 (Damodar)>         %
% Author: Damodar Rajbhandari (dphysicslog@gmail.com) %
% Site: physicslog.com and damodarrajbhandari.com.np  %
%%%%%%%%%%%%%%%%%%%%%%%%%%%%%%%%%%%%%%%%%%%%%%%%%%%%%%%

\documentclass[12pt,a4paper]{article}

\usepackage[utf8]{inputenc} %utf 8 encoding

\usepackage{amsmath} % for ams style maths
\usepackage{amsfonts} % for ams style fonts
\usepackage{amssymb}  % for ams mathematical symbols

\usepackage[a4paper]{geometry} % To adjust the margin of the page
\geometry{left = 1in, right= 1in, top = 1.25in, bottom = 1.25in, headheight= 0.5in} %Rescaling the Margin

\usepackage{graphicx} % To insert graphics in the document
\graphicspath{{figures/}{../figures/}} % Justify the graphics path

\usepackage{hyperref} % To make hyperlink

\usepackage{indentfirst} % makes indentation in the starting of every paragraph

\title{Notes for the Physics GRE to Ace} % Title of the document
\author{Damodar Rajbhandari \\ \href{https://damodarrajbhandari.com.np}{damodarrajbhandari.com.np} \\ \href{https://www.physicslog.com}{physicslog.com}} % Author of this document

\begin{document}

\maketitle % Retrieving the title informations like title, author and date.

pGRE stands for Physics subject test for Graduate Record Examination (GRE). This test is designed and taken by Educational Testing Service (ETS). The test consists of 100 questions with 5 choices, and you have 2 hours and 50 minutes to do it. This means, you only have an average of 1.7 minutes per problem. This is why many test takers found it's difficult. But, I want to note here from the very beginning, there are some tricks that makes calculations much simpler and which do not require special knowledge. 

Unlike general GRE and TOEFL, pGRE is a paper-based test. And, it yields a total score on a 200 to 900 score scale in 10-point increments. Most of the questions can be answered on the basis of how much knowledge and understanding  do you have on your first three years of undergraduate physics. You must use International system (SI) of units. You are not allowed to use calculator during the test. But, the test center will provide you a table sheet of information representing various physical constant and a few conversion factors, log book, pencils, and test book along with the bubble answer sheets. Do not worry about you cannot do calculations without calculator because it is enough to know or guess an answer out of five from your approximate numerical value. Even though pGRE will test your physics solving brilliance but it may also test your ability on certain mathematical method and their applications in physics.  We are lucky that nothing is subtracted from a score if you answer a question incorrectly or leave it blank. But, in the past, they used to penalize incorrect answer by subtracting one quarter per incorrect answer from your score.

The pGRE is offered 3 times per year, in April, September, and October. Most people prefer to take pGRE either on September or October. It's up to you. The cost of the test is 150\$ all around the world. You can send four scores for free after you gave your test. But in TOEFL, you can add/delete/modify your score recipients after you register for that test. For an additional score report per recipient, you need to pay 27\$. You can view your scores online from your ETS GRE account for free. The reason we need to send our score to the University (i.e. score recipient) in-order to authorize our score. This means if you send your score by your email account or uploading score's pdf from application portal, the admission committee will consider it as unofficial. Your scores can be reportable for five years following your test date. For example, scores for a test taken on October 27, 2018, are reportable through October 26, 2023. If you want to register for pGRE from Nepal, you need to go to Nabil Bank either at Teendhara (Durbar Marg) branch or Chabahil branch, Kathmandu. Please make sure you have created an ETS account for GRE along with your personal information.


Let me give you some insights of the pGRE syllabus by ETS, and notes that I have taken from different resources. Lets get started!

%%%%%%%%%%%%%%%%%%%%%%%%%%%%%%%%%%%%%%%%%%%%%%%%%%%%%%%%%%%%%%%%%%%%%%%%%%%
\section{Classical Mechanics (20\%)}

It includes kinematics, Newton's laws, work and energy, oscillatory motion, rotational motion about a fixed axis, dynamics of systems of particles, central forces and celestial mechanics, three-dimensional particle dynamics, Lagrangian and Hamiltonian formalism, non-inertial reference frames, and elementary topics in fluid dynamics.

\subsection{Kinematics}

\subsubsection{Linear Motion in the inertial frame of reference}

\noindent\textbf{Average velocity}
\begin{align}
v = \frac{\Delta x}{\Delta t} = \frac{x_{2} - x_{1}}{t_{2} - t_{1}}
\end{align}
\\
\textbf{Instantaneous velocity}
\begin{align}
v = \lim_{\Delta t \to 0} \frac{\Delta x}{\Delta t} = \frac{dx}{dt} = v(t)
\end{align}
\\
\textbf{Kinematic equations of motion under constant acceleration: SUVAT rule}
\begin{align}
v &= u + at\\
(x - x_{0}) = s &= ut + \frac{1}{2}at^{2} \\
v^{2} &= u^{2} + 2as \\
(x - x_{0}) &= \frac{1}{2}(u + v)t 
\end{align}

\subsubsection{Uniform circular motion}

\noindent\textbf{Centripetal acceleration}
\begin{align}
a = \frac{v^{2}}{r}
\end{align}
\\
\textbf{Angle}
\begin{align}
\theta = \frac{\text{arc}}{\text{radius}}
\end{align}
\\
\textbf{Angular velocity}
\begin{align}
\omega = \lim_{\Delta t \to 0}\frac{\Delta \theta}{\Delta t}
\end{align}
\\
\textbf{Relation of v and $\omega$}
\begin{align}
v = r\omega
\end{align}
\\
\textbf{Relation of a and $\omega$}
\begin{align}
a = r\omega^{2}
\end{align}
\\
\textbf{Angular acceleration}
\begin{align}
\alpha = \lim_{\Delta t \to 0}\frac{\Delta \omega}{\Delta t}
\end{align}
\\
\textbf{Rotational equation of motion under constant angular acceleration}
\begin{align}
\omega &= \omega_{0} + \alpha t \\
\theta &= \omega_{0}t + \frac{1}{2}\alpha t^{2} \\
\omega^{2} &= \omega_{0}^{2} + 2\alpha \theta \\
\theta &= \frac{1}{2}(\omega_{0} + \omega)t
\end{align}

\subsection{Newton's laws of motion}

\noindent\textbf{First law:} A body continues in its state of rest or \emph{uniform motion} unless any external \emph{unbalanced force} act upon it.
\\
\textbf{Second law:} The net force \emph{on} a body is proportional to its rate of change of \emph{linear momentum}.
\begin{align}
F = \frac{dp}{dt} = ma
\end{align}
\\
\textbf{Third law:} When a particle A exerts a force on another particle B (i.e. $F_{AB}$), B simultaneously exerts a force on A (i.e. $F_{BA}$) with the \emph{same magnitude} in the \emph{exactly opposite direction}.
\begin{align}
F_{AB} = - F_{BA}
\end{align}

\subsection{Confused words in the Classical Mechanics}

\begin{itemize}
\item \emph{String} (i.e. straight wire or thread) and \emph{Spring} (i.e. a helical metal coil).
\item \emph{Gravity} (i.e. gives the answer why every objects fall towards the earth) and \emph{Cavity} (i.e. a hollow space inside solid or liquid)
\end{itemize}
%%%%%%%%%%%%%%%%%%%%%%%%%%%%%%%%%%%%%%%%%%%%%%%%%%%%%%%%%%%%%%%%%%%%%%%%%%%%%%
\section{Electromagnetism (18\%)}

It includes electrostatics, current and DC circuits, magnetic fields in free space, Lorentz force, induction, Maxwell's equations and their applications, electromagnetic waves, AC circuits, and magnetic and electric fields in matter.

%%%%%%%%%%%%%%%%%%%%%%%%%%%%%%%%%%%%%%%%%%%%%%%%%%%%%%%%%%%%%%%%%%%%%%%%%%%%%%
\section{Quantum Mechanics (12\%)}

It includes fundamental concepts, solutions of the Schr\"{o}dinger equation (like square wells, harmonic oscillators and hydrogen atoms), spin, angular momentum, wave function, symmetry, and elementary perturbation theory.


%%%%%%%%%%%%%%%%%%%%%%%%%%%%%%%%%%%%%%%%%%%%%%%%%%%%%%%%%%%%%%%%%%%%%%%%%%%%%%
\section{Thermodynamics and Statistical Mechanics (10\%)}

It includes the laws of thermodynamics, thermodynamic processes, equations of state, ideal gases, kinetic theory, ensembles, statistical concepts, calculation of thermodynamics quantities, thermal expansion, and heat transfer.


%%%%%%%%%%%%%%%%%%%%%%%%%%%%%%%%%%%%%%%%%%%%%%%%%%%%%%%%%%%%%%%%%%%%%%%%%%%%%%
\section{Atomic Physics (10\%)}

It includes properties of electrons, Bohr's model, energy quantization, atomic structure, atomic spectra, selection rules, black-body radiation, X-rays, atoms in electric fields, and atoms in magnetic fields.


%%%%%%%%%%%%%%%%%%%%%%%%%%%%%%%%%%%%%%%%%%%%%%%%%%%%%%%%%%%%%%%%%%%%%%%%%%%%%%
\section{Optics and wave phenomena (9\%)}

It includes wave properties, superposition, interference, diffraction, geometrical optics, polarization, and Doppler effect.


%%%%%%%%%%%%%%%%%%%%%%%%%%%%%%%%%%%%%%%%%%%%%%%%%%%%%%%%%%%%%%%%%%%%%%%%%%%%%%
\section{Specialized topics (9\%)}

It includes Nuclear physics (e.g., nuclear properties, radioactive decay, fission reactions, and fusion reactions), Particle physics (e.g., fundamental properties of elementary particles), Condensed Matter physics (e.g., crystal structure, x-ray diffraction, thermal properties, electron theory of metals, semiconductors, and superconductors), and Miscellaneous (e.g., Astrophysics, Mathematical methods, and Computer applications). 

\subsection{Mathematical methods}

It includes single and multivariate calculus, co-ordinate systems (rectangular, cylindrical and spherical), vector algebra and vector differential operators, Fourier series, partial differential equations, boundary value problems, matrices and determinants, and function of complex variables.


%%%%%%%%%%%%%%%%%%%%%%%%%%%%%%%%%%%%%%%%%%%%%%%%%%%%%%%%%%%%%%%%%%%%%%%%%%%%%%
\section{Special Relativity (6\%)}

It includes introductory concepts of relativity, simultaneity, energy and momentum, four-vectors, Lorentz transformation, velocity addition, time dilation, and length contraction.

\section{Laboratory methods (6\%)}

It includes data and error analysis, electronics, instrumentation, radiation detection, counting statistics, interaction of charged particles with matter, lasers and optical interferometers, dimensional analysis, and fundamental applications of probability and statistics.


\end{document}